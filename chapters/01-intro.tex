\chapter{Introduction}
\label{sec:intro}

Fifty years after its initial proposal by Edgar F. Codd,
 the relational model has cemeted its status as the de-facto data model 
 in nearly all modern databases.
It provides data independence 
 which has allowed data systems to scale to unprecedented scales,
 while guaranteeing performance and correctness.
The success of the relational model is witnessed 
 by the popularity of SQL 
 which is uniquitous in computing systems 
 of any size from smartphones to data centers.

 Nevertheless, traditional relational databases are struggling 
  to meet the demands of modern data analytics.
Today's data analytics involve new kinds of data, 
 as well as new kinds of computation over such data. 
For example, machine learning workloads involve linear algebra operations
 running over data stored as matrices. 
For another example, graph analytics require iterative algorithms
 over graph data.
Running these workloads in existing relational databases 
 is both cumbersome 
 (because SQL is a poor language for expressing the computation),
 and slow (because the systems are not optimized for such workloads).

This dissertation is motivated by the following question: 
{\em How can we renew relational database systems 
 to support modern data analytics?}
In this dissertation, I propose a new language 
 for writing relational programs;
 a new algorithm for the relational join;
 and techniques to optimize relational queries. 

\section{Motivation}
\label{sec:intro:motivation}

There are tons of specialized systems for the tasks 
 in modern data analytics.
For example (list machine learning systems), and (list graph databases).
Given the abundance of such options,
 why do we still care about the relational systems?
It's the same reason that made relational successful 
 for traditional data management: data independence.
At a high level, data independence decouples
 how computation is specified from how it is carried out. 
More specifically, the programmer writes a high-level,
 declarative program {\em once},
 and the relational system automatically 
 compiles and optimizes the program to execute 
 in a variety of environments:
 over data in-memory or on-disk;
 the data may be distributed over several machines,
 or it may arrive in a stream;
 and the system may take advantage of a large number of parallel threads,
 or specialized accelerators like the GPU.
Other concerns include indexing, view materialization,
 approximation, and fault-toleration.
In short, the users of a relational database need only focus on
 specifying {\em what} to compute, 
 and the database systems automatically figures out {\em how}. 

Database systems have been roughly catergorized as {\em transactional},
 or OLTP, and {\em analytic}, or OLAP.
In this dissertation we focus on the latter category.
The main objective of the user of an OLAP database 
 is to analyze the stored data to gain insight about it.
For example, the owner of an online store may hope to 
 analyze historic sales data to understand 
 the affect of promotion campaigns.
An OLAP query often touches large amount of data,
 performing complex operations over them.
In other words, we run big queries over big data. 

Over the past decade, the landscape of data analytics has changed.
On one hand, people are working with many new forms of data, 
 in the form of graphs, matrices and tensors.
On the other, they are running new kinds of computation 
 over such data, including linear/tensor algebra operations 
 and iterative graph algorithms.
Initially designed and optimized for loop-free queries over tabular data,
 relational databases struggle to support modern data analytics. 
They struggle in two levels: 
 at the interface between the system and the human,  
 the SQL language is not designed to express linear algebra operations 
 and iteration. 
SQL queries expressing such computation are complex,
 brittle and ugly.
At the interface between the system and the machine,
 traditional systems are not optimized for the workloads 
 central to modern analytics.
For example, linear algebra extensively uses aggregation 
 which is rarely optimized in existing databases.
For another example, most graph algorithms are iterative,
 yet existing databases only provide na\"ive optimizations for iterations. 
Namely, most systems optimize within each iteration independently;
 the few systems that do optimize across iterations 
 only implement classic techniques like the magic-set transformation. 

Facing these challenges, a relational system modernized 
 for today's data analytics 
 should therefore provide an ergonomic language 
 for specifying a variety of computation 
 over different forms of data,
 and at the same time optimize it for efficient execution. 

\section{Contributions}
\label{sec:intro:contributions}

Make intro accessible, expand the section for someone who doesn't follow 
 the work. 

\section{Related Work}
\label{sec:intro:related-work}

Include miniKanren. Prolog v.s. Datalog: top-down v.s.
bottom-up. Left recursion doesn't work in prolog. 