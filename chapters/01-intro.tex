\chapter{Introduction}
\label{sec:intro}

Fifty years after its initial proposal by Edgar F. Codd,
 the relational model has cemeted its status as the de-facto data model 
 in nearly all modern databases.
It provides {\em data independence}, 
 which has allowed data systems to scale to unprecedented sizes
 while guaranteeing performance and correctness.
The success of the relational model is witnessed 
 by the popularity of SQL, 
 uniquitous in computing systems 
 from smartphones to data centers.

 Nevertheless, traditional relational databases are struggling 
  to meet the demands of modern data analytics.
Today's data analytics involve new kinds of data, 
 as well as new kinds of computation over such data. 
For example, machine learning workloads involve linear algebra operations
 running over data stored as matrices. 
For another example, graph analytics require iterative algorithms
 over graph data.
Running these workloads in existing relational databases 
 is both cumbersome and slow:
 SQL is a poor language for expressing the computation
 and the systems are not optimized for such workloads.

This dissertation is motivated by the question: 
{\em How can we renew relational database systems 
 to support modern data analytics?}
Towards that end, I propose a new language foundation
 for writing relational programs;
 a new algorithm for the relational join;
 and techniques to optimize relational queries. 
Together, these three ingredients make up 
 the basis for a new generation of relational systems
 that are more expressive and more efficient.
Using these systems,
 the analyst may author entire application programs
 instead of simple queries {\em relationally},
 entering the era of {\em relational programming}.

\section{Motivation and Contributions}
\label{sec:intro:motivation}

% For fifty years, the relational data model has been the main choice for
% representing, modeling, and processing data.  
The main query language for relational databases, SQL, 
 is found today in a wide range of applications and
 devices. 
%  from smart phones, to database servers, to distributed
%  cloud-based clusters.
The reason for its success is the {\em data
  independence principle}, which separates the declarative model from
the physical implementation~\cite{DBLP:journals/cacm/Codd70}, and
enables advanced implementation techniques, such as cost-based
optimizations, indices, materialized views, incremental view
maintenance, parallel evaluation, and many others, while keeping the
same simple, declarative interface to the data unchanged.

But analysts today often need to perform tasks that require
iteration over the data.
Gradient descent, clustering, page-rank, network centrality, inference
in knowledge graphs are some examples of common tasks in data science
that require iteration.  While SQL has introduced recursion since 1999
(through Common Table Expressions, CTE), it has many cumbersome
limitations and is little used in practice~\cite{frankmcsherry-2022}.

The need to support recursion in a declarative language led to the
development of Datalog in the mid 80s~\cite{DBLP:conf/pods/Vianu21}.
Datalog adds recursion to the relational query language, yet enjoys several elegant
properties: it has a simple, declarative semantics; its na\"ive
bottom-up evaluation algorithm always terminates; and it admits a few
powerful optimizations, such as semi-na\"ive evaluation and magic set
rewriting.  Datalog has been extensively studied in the literature;
see~\cite{DBLP:journals/ftdb/GreenHLZ13} for a survey
and~\cite{DBLP:books/mc/18/MaierTKW18,DBLP:conf/pods/Vianu21} for historical notes.
We will also briefly review the semantics and execution of Datalog in
 Chapter~\ref{chap:background}.

However, Datalog is not the answer to modern needs, because it only
supports monotone queries over sets.  Most tasks in data science today
require the interleaving of recursion and aggregate computation.
Aggregates are not monotone under set inclusion, and therefore they
are not supported by the framework of pure Datalog.  Neither SQL'99
nor popular open-source Datalog systems like
Souffl\'e~\cite{DBLP:conf/cav/JordanSS16} allow recursive queries to
have aggregates.  While several proposals have been made to extend
Datalog with
aggregation~\cite{DBLP:conf/pods/GangulyGZ91,DBLP:conf/pods/RossS92,DBLP:journals/jcss/GangulyGZ95,DBLP:journals/vldb/MazuranSZ13,DBLP:conf/icde/ShkapskyYZ15,DBLP:conf/sigmod/ShkapskyYICCZ16,DBLP:conf/amw/ZanioloYDI16,DBLP:journals/tplp/ZanioloYDSCI17,DBLP:conf/amw/ZanioloYIDSC18,DBLP:journals/tplp/CondieDISYZ18,DBLP:conf/sigmod/0001WMSYDZ19,DBLP:journals/corr/abs-1910-08888,DBLP:journals/corr/abs-1909-08249,DBLP:journals/debu/ZanioloD0LL021},
these extensions are at odds with the elegant properties of Datalog
and have not been adopted by either Datalog systems or SQL engines.

\textbf{
The first contribution of this dissertation is a foundation for a query language that
supports both recursion and aggregation.}  
Our language, called \datalogo, 
 retains many of the elegant properties of Datalog,
 while extending its expressiveness to support aggregation.
Our proposal is based on the
concept of $K$-relations, introduced in a seminal
paper by Green, Karvounarakis, and Tannen~\cite{DBLP:conf/pods/GreenKT07}.
In a $K$-relation, tuples are
mapped to a fixed semiring. Standard relations (sets) are
$\B$-relations where tuples are mapped to the Boolean semiring $\B$,
relations with duplicates (bags) are $\N$-relations, sparse tensors
are $\R$-relations, and so on.  Queries over $K$-relations are the
familiar relational queries, where the operations $\wedge, \vee$ are
replaced by the operations $\otimes, \oplus$ in the semiring;
importantly, an existential quantifier $\exists$ becomes an
$\oplus$-aggregate operator.
$K$-relations are a very powerful abstraction, because they open up
the possibility of adapting query processing and optimization
techniques to other domains~\cite{DBLP:conf/pods/KhamisNR16}.

To evaluate any relational program, 
 including \datalogo programs,
 classic Datalog programs, 
 or even SQL queries without recursion, 
 the key operation is the relational join.
The join allows us to combine data from different sources, 
 as well as compose computation from multiple programs.
Most mainstream databases today evaluate a query 
 by joining two relations at a time.
In other words, they implement {\em binary join} algorithms.
Over the last decade, worst-case optimal join (\WCOJ)~\cite{
  DBLP:conf/pods/NgoPRR12,
  DBLP:conf/icdt/Veldhuizen14, 
  DBLP:journals/sigmod/NgoRR13, 
  DBLP:conf/pods/000118}
 has emerged as
 a breakthrough in the design of efficient join algorithms.  
It can be
asymptotically faster than traditional binary joins, all the while
remaining simple to understand and
implement~\cite{DBLP:journals/sigmod/NgoRR13}.  
\WCOJ has opened up a flourishing field of research, leading to both theoretical
results~\cite{DBLP:journals/sigmod/NgoRR13,DBLP:conf/pods/Khamis0S17}
and practical
implementations~\cite{DBLP:conf/icdt/Veldhuizen14,DBLP:journals/tods/AbergerLTNOR17,DBLP:journals/pvldb/FreitagBSKN20, DBLP:journals/pvldb/MhedhbiS19}.
However, traditional binary join algorithms have benefited from 
  decades of research and engineering.
Techniques like column-oriented layout, vectorization, 
  and query optimization
  have contributed compounding constant-factor speedups,
  making it challenging for \WCOJ to be competitive in practice.

\textbf{
The second contribution of this dissertation is a new join algorithm,
  called \FJ, that unifies \WCOJ and binary join.}
We propose several new techniques to make \FJ outperform
both binary join and \WCOJ:
\begin{enumerate}
\item An algorithm to convert any binary join plan to a \FJ
  plan that runs as fast or faster.
\item A new data structure called
\COLT (for \emph{Column-Oriented Lazy Trie}), adapting the classic
column-oriented layout to improve the trie data structure used in
\WCOJ.
\item A vectorized execution algorithm for \FJ.
\end{enumerate}

\section{Related Work}
\label{sec:intro:related-work}

Include miniKanren. Prolog v.s. Datalog: top-down v.s.
bottom-up. Left recursion doesn't work in prolog. 