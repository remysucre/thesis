\chapter{Conclusions and Future Outlook}
\label{chap:conclusions}

In this dissertation we have enhanced the relational paradigm 
 of data processing to support more complex data and programs.
We first extended the traditional Datalog 
 language with support for recursive aggregates,
 enabling the expression of a wide range of data processing tasks.
We then described a new algorithm to speed up the relation join, 
 a key operation in relational data processing.
We presented techniques to optimize linear algebra programs 
 through a relational lens,
 and finally showed how to optimize recursive programs over relational data.
Our approach follows the core principle of the relational model:
 to allow the programmer to specify computation in a high-level, declarative language,
 and let the system optimize the program for efficient execution.

Nevertheless, to realize the full potential of the relational model 
 in modern data processing, 
 there are still many challenges to overcome.
Namely, the relational model offers {\em data independence},
 meaning the programmer needs only to write the program logic once, 
 and the system will automatically handle
 optimization, parallelism, distribution, incremetal maintenance, 
 persistence, fault tolerance, etc.
Addressing each of these challenges in the context of modern data processing
 is a research topic of its own right.
In the following, we point out a few promising directions 
 that will bring us closer towards our vision of relational programming.

\paragraph*{Scheduling Languages.}
Building a relational data system is hard.
Half a century after the relational model was first proposed,
 we are still perfecting the arts and crafts of non-recursive
 simple queries over tabular data.
We believe the relational future of programming is much more tangible, 
 if we rethink the relationship between the data analyst and the data system.
In a perfect world, the analyst need only specify the program logic 
 in the highest possible level of abstraction, 
 and the system will fill in all the details automatically.
Building such a fully autonomous system is indeed a daunting task.
The rise of {\em scheduling languages} like Halide~\cite{DBLP:conf/pldi/Ragan-KelleyBAPDA13} 
 in domain-specific computing provides a more pragmatic alternative.
The programmer may start with a simple, yet inefficient, program that 
 is already executable (perhaps for prototyping purposes).
Then, instead of relying on the system to automatically optimize their program, 
 the programmer {\em guides} the system to gradually refine the program to be more efficient.
For example, one may instruct the system to unroll or fuse certain loops, 
 or arrange data into tiles to take advantage of SIMD instructions.
In fact, some database systems already provide mechanisms for 
 the programmer to specify hints to the optimizer,
 and the most performance-critical queries are already 
 decorated with hand-picked join orders, cardinality estimates, and index selections.
We believe the first practical, high-performance relational programming system 
 will be built on top of such a scheduling language.

\paragraph*{Modular Architecture.}
Traditionally, database systems are built as monolithic systems.
Partly due to the commercial nature of many database systems, 
 and partly due to the rapid evolution of hardware and software,
 each system is built from scratch 
 around a new architectural design.
This means building every new system requires significant effort
 that is not reusable by other systems.
However, the design of many components of a database system 
 has largely stablized:
 query optimizers usually follow the \textsf{CASCADE} framework~\cite{Graefe95a},
 with the execution engine implementing either the pull-based or push-based model~\cite{DBLP:journals/pvldb/KerstenLKNPB18},
 and the data are stored as either rows or columns~\cite{DBLP:journals/ftdb/AbadiBHIM13}.
Even within the optimizer, most systems implement variants of the dynamic programming algorithm~\cite{DBLP:conf/vldb/MoerkotteN06}
 which interacts with a cardinality estimator to find the cheapest join order.
We therefore advocate developing modular, reusable components 
 for relational data systems, 
 and designing new system architectures by composing these components.
There is already encourage progress in this direction, 
 for example the Apache Calcite project~\cite{begoli2018apache} and 
 the Orca optimizer~\cite{DBLP:conf/sigmod/SolimanAREGSCGRPWNKB14}.
Again taking inspireation from the programming languages community, 
 the database community should aim to build a collection of 
 database components like the LLVM project did for compilers~\cite{DBLP:conf/cgo/LattnerA04}.