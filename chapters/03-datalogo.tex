\chapter{Datalog Chapter}
\label{chap:datalogo}

For fifty years, the relational data model has been the main choice for
representing, modeling, and processing data.  Its main query
language, SQL, is found today in a wide range of applications and
devices, from smart phones, to database servers, to distributed
cloud-based clusters.  The reason for its success is the {\em data
  independence principle}, which separates the declarative model from
the physical implementation~\cite{DBLP:journals/cacm/Codd70}, and
enables advanced implementation techniques, such as cost-based
optimizations, indices, materialized views, incremental view
maintenance, parallel evaluation, and many others, while keeping the
same simple, declarative interface to the data unchanged.

But scientists today often need to perform tasks that require
iteration over the data.
Gradient descent, clustering, page-rank, network centrality, inference
in knowledge graphs are some examples of common tasks in data science
that require iteration.  While SQL has introduced recursion since 1999
(through Common Table Expressions, CTE), it has many cumbersome
limitations and is little used in practice~\cite{frankmcsherry-2022}.

The need to support recursion in a declarative language led to the
development of {\em datalog} in the mid 80s~\cite{DBLP:conf/pods/Vianu21}.
Datalog adds recursion to the relational query language, yet enjoys several elegant
properties: it has a simple, declarative semantics; its na\"ive
bottom-up evaluation algorithm always terminates; and it admits a few
powerful optimizations, such as semi-na\"ive evaluation and magic set
rewriting.  Datalog has been extensively studied in the literature;
see~\cite{DBLP:journals/ftdb/GreenHLZ13} for a survey
and~\cite{DBLP:books/mc/18/MaierTKW18,DBLP:conf/pods/Vianu21} for historical notes.

However, datalog is not the answer to modern needs, because it only
supports monotone queries over sets.  Most tasks in data science today
require the interleaving of recursion and aggregate computation.
Aggregates are not monotone under set inclusion, and therefore they
are not supported by the framework of pure datalog.  Neither SQL'99
nor popular open-source datalog systems like
Souffl\'e~\cite{DBLP:conf/cav/JordanSS16} allow recursive queries to
have aggregates.  While several proposals have been made to extend
datalog with
aggregation~\cite{DBLP:conf/pods/GangulyGZ91,DBLP:conf/pods/RossS92,DBLP:journals/jcss/GangulyGZ95,DBLP:journals/vldb/MazuranSZ13,DBLP:conf/icde/ShkapskyYZ15,DBLP:conf/sigmod/ShkapskyYICCZ16,DBLP:conf/amw/ZanioloYDI16,DBLP:journals/tplp/ZanioloYDSCI17,DBLP:conf/amw/ZanioloYIDSC18,DBLP:journals/tplp/CondieDISYZ18,DBLP:conf/sigmod/0001WMSYDZ19,DBLP:journals/corr/abs-1910-08888,DBLP:journals/corr/abs-1909-08249,DBLP:journals/debu/ZanioloD0LL021},
these extensions are at odds with the elegant properties of datalog
and have not been adopted by either datalog systems or SQL engines.

In this paper we propose a foundation for a query language that
supports both recursion and aggregation.  Our proposal is based on the
concept of $K$-relations, introduced in a seminal
paper~\cite{DBLP:conf/pods/GreenKT07}.  In a $K$-relation, tuples are
mapped to a fixed semiring. Standard relations (sets) are
$\B$-relations where tuples are mapped to the Boolean semiring $\B$,
relations with duplicates (bags) are $\N$-relations, sparse tensors
are $\R$-relations, and so on.  Queries over $K$-relations are the
familiar relational queries, where the operations $\wedge, \vee$ are
replaced by the operations $\otimes, \oplus$ in the semiring;
importantly, an existential quantifier $\exists$ becomes an
$\oplus$-aggregate operator.
$K$-relations are a very powerful abstraction, because they open up
the possibility of adapting query processing and optimization
techniques to other domains~\cite{DBLP:conf/pods/KhamisNR16}.

Our first contribution is to introduce an extension of datalog to
$K$-relations.  We call the language $\name$, where the superscript
$o$ represents a (semi)-ring.  $\Name$ has a declarative semantics
based on the least fixpoint, and supports both recursion and
aggregates.  We illustrate throughout this paper its utility through
several examples that are typical for recursive data processing.  In
order to define the least fixpoint semantics of $\name$, the semiring
needs to be partially ordered.  For this purpose, we introduce an
algebraic structure called a {\em Partially Ordered Pre-Semiring (POPS)\/},
which generalizes the more familiar naturally ordered semirings.  This
generalization is necessary for some applications.  For example, the
bill-of-material program (Example~\ref{ex:sum1:sum2}) is naturally
expressed over the lifted reals, $\R_\bot$, which is a POPS that is
not naturally ordered.

Like datalog, $\name$ can be evaluated using the {\em na\"ive algorithm},
by repeatedly applying all rules of the program, until there is no
more change.  However, unlike datalog, a $\name$ program may diverge.
Our second contribution consists of a full characterization of the
POPS that guarantee that every $\name$ program terminates.  More
precisely, we show that termination is guaranteed iff the POPS enjoys
a certain algebraic property called {\em
  stability}~\cite{semiring_book}.  The result is based on an analysis
of the fixpoint of a vector-valued polynomial function over a semiring, which is of
independent interest.  With some abuse, we will say in this paper that
a $\name$ program {\em converges}, if the na\"ive algorithm terminates
in a finite number of steps; we do not consider ``convergence in the
limit'', for example in an $\omega$-continuous
semirings~\cite{DBLP:conf/pods/GreenKT07,DBLP:journals/jacm/EsparzaKL10}.

Finally, we describe how the {\em semi-na\"ive algorithm} can be
extended to $\name$, under certain restrictions on the POPS.  This
should be viewed as an illustration of the potential for applying
advanced optimizations to $\name$: in a companion
paper~\cite{DBLP:conf/sigmod/WangK0PS22}, we introduced a simple, yet
powerful optimization technique for $\name$, and showed, among other
things, that magic set rewriting can be obtained using several
applications of that rule.

At its essence, a $\name$ program consists of solving a fixpoint
equation in a semiring, which is a problem that was studied in a variety
of areas, like automata theory, program analysis, and many
others~\cite{MR1470001,DBLP:conf/popl/CousotC77,MR1728440,MR1059930,
  DBLP:conf/lics/HopkinsK99, DBLP:journals/tcs/Lehmann77,
  semiring_book,MR609751}.  The existence of the fixpoint is usually
ensured by requiring the semiring to be $\omega$-continuous. For
example, Green et al.~\cite{DBLP:conf/pods/GreenKT07} studied the
provenance of datalog on $K$-relations, while Esparza et
al.~\cite{DBLP:journals/jacm/EsparzaKL10} studied dataflow equations,
in both cases requiring the semiring to be $\omega$-continuous.  This
guarantees that the least fixpoint exists, even if the na\"ive algorithm
diverges.

In addition to the (semi-)na\"ive method, which is a first-order method for solving
fixpoint equations, there is a second-order method called the {\em Newton's method},
discovered relatively recently \cite{DBLP:journals/jacm/EsparzaKL10,DBLP:conf/lics/HopkinsK99}
in the program analysis literature.
It was shown that Newton's method requires a smaller number of iterations than the
na\"ive algorithm.\footnote{The na\"ive algorithm is called Kleene iteration
method in~\cite{DBLP:journals/jacm/EsparzaKL10}.} In the particular
case of a commutative and idempotent semiring, Newton's method always converges, while the
na\"ive algorithm may diverge~\cite{DBLP:journals/jacm/EsparzaKL10}.
However, every iteration of Newton's method requires solving a least fixpoint solution
to an inner-recursion, which is expensive.
Here, we are seeing the exact same phenomenon when comparing gradient descent vs.
Newton's method in continuous optimization: while Newton's method may converge in fewer
steps, every step is more expensive, and requires the materialization of a large intermediate
result (the Hessian matrix).
In continuous optimization, Newton's method is rarely used for large-scale problems for
this reason~\cite{DBLP:books/sp/NocedalW99}.
For $\name$, it remains unclear whether Newton's method is more efficient in practice than
the na\"ive algorithm.  One experimental evaluation~\cite{DBLP:conf/popl/RepsTP16} has
found that it is not. In contrast, the na\"ive algorithm, and its
extension to the semi-na\"ive algorithm, is simple, and implemented by
all datalog systems today.  In this paper we focus only on the
convergence of the na\"ive algorithm, and do not consider Newton's
method.


\section{Overview of the Results}
\label{subsec:overview}

We present here a high-level overview of the main results in this
paper.  We start by recalling the syntax of a (traditional) datalog
program $\Pi$, which consists of a set of rules of the form:
%
\begin{align}
  R_0(\bm X_0) &\cd R_1(\bm X_1) \wedge \cdots \wedge R_m(\bm X_m) \label{eq:datalog}
\end{align}
%
where $R_0, \ldots, R_m$ are relation names (not necessarily distinct)
and each $\bm X_i$ is a tuple of variables and/or constants.  The atom
$R_0(\bm X_0)$ is called the head, and the conjunction
$R_1(\bm X_1) \wedge \cdots \wedge R_m(\bm X_m)$ is called the body.
A relation name that occurs in the head of some rule in $\Pi$ is
called an {\em intensional database predicate} or IDB, otherwise it is
called an {\em extensional database predicate} or EDB.  The EDBs form
the input database, which is denoted $I$, and the IDBs represent the
output instance computed by $\Pi$, which we denote by $J$.  The finite
set of all constants occurring in $I$ is called the {\em active
  domain} of $I$, and denoted $\adom(I) \subseteq D$.
%
The textbook example of a datalog program is one that computes the
transitive closure of a graph defined by the edge relation $E(X,Y)$:
%
\begin{align*}
  T(X,Y) &\cd E(X,Y) \\
  T(X,Y) &\cd T(X,Z) \wedge E(Z,Y)
\end{align*}
%
Here $E$ is an EDB predicate, and $T$ is an IDB predicate.  Multiple
rules with the same head are interpreted as a disjunction, and in this
paper we prefer to combine them in a single rule, with an explicit
disjunction.  In particular, the program above becomes:
%
\begin{align}
  T(X,Y) & \cd E(X,Y) \vee \exists_Z (T(X,Z) \wedge E(Z,Y)) \label{eq:datalog:intro}
\end{align}
%
The semantics of a datalog program is the least fixpoint of the
function defined by its rules, and can be computed by the {\em na\"ive
  evaluation algorithm}, as follows: initialize all IDBs to the empty
set, then repeatedly apply all rules, updating the state of the IDBs,
until there is no more change to the IDBs.  The algorithm always
terminates in time polynomial in $\adom(I)$.

We introduce $\name$, a language that generalize datalog to relations
over a {\em partially ordered, commutative pre-semiring}, or POPS.  A
POPS, $\bm P$, is an algebraic structure with operations
$\oplus, \otimes$ on a domain $P$, which is also a partially ordered set with an order
relation $\sqsubseteq$ and a smallest element $\bot$. (Formal
definition is in Sec.~\ref{sec:pops}).  For example, the Boolean
semiring $\B$ forms a POPS where $(\oplus,\otimes) = (\vee, \wedge)$, over domain
$\{0,1\}$, and the order relation is $0 \leq 1$.  For another example, the {\em
  tropical semiring}, $\trop$, consists of the real numbers $\R$, the
operations are $\min, +$, and the order relation $x \sqsubseteq y$ is
the reverse of the usual order $x \geq y$.  If
we restrict the tropical semiring to the non-negative reals, $\R_+$, then
we denote the POPS by $\trop^+$.  For a fixed POPS $\bm P$, a
$\bm P$-relation of arity $k$ is a function that maps all $k$-tuples
over some domain to values in $\bm P$; for example, a standard
relation is a $\B$-relation, where $\B$ is the Boolean semiring, while
a relation with duplicates (a bag) is an $\N$-relation.  A $\name$
program consists of a set of rules, e.g.
like~\eqref{eq:datalog:intro}, where all relations are
$\bm P$-relations, and the operators $\wedge, \vee, \exists_Z$ are
replaced by $\otimes, \oplus, \bigoplus_Z$, and its semantics is
defined as its least fixpoint, when it exists, or undefined otherwise.
We will illustrate the utility of $\name$ through multiple examples in
this paper, and give here only a preview.

\begin{ex} \label{ex:intro} The {\em all-pairs shortest paths} (APSP) problem is to compute the
  shortest path length $T(X,Y)$ between all pairs $X,Y$ of vertices in
  the graph, given that $E(X,Y)=$ the length of the edge $(X,Y)$ in
  the graph.  We assume that both $E$ and $T$ are $\trop^+$-relations.
  The APSP problem can be expressed very compactly in $\name$ as
  %
  \begin{align}
    T(X,Y) &\cd E(X,Y) \oplus \bigoplus_Z T(X,Z) \otimes E(Z,Y), \label{eqn:linear:tc}
  \end{align}
%
  where $(\oplus,\otimes) = (\min, +)$ are the ``addition'' and
  ``multiplication'' in $\trop^+$.  If we instantiate these operations
  in~\eqref{eqn:linear:tc}, the program becomes:
  %
  \begin{align}
    T(X,Y) &\cd \min\left(E(X,Y), \min_Z (T(X,Z) + E(Z,Y))\right), \label{eqn:apsp}
  \end{align}
%
  If we use a different POPS in~\eqref{eqn:linear:tc}, then the same
  $\name$ program will express similar problems, in exactly the same
  way. For example, if the relations $E, T$ are $\B$-relations, then
  the program~\eqref{eqn:linear:tc} becomes the {\sf transitive
    closure} program~\eqref{eq:datalog:intro}; alternatively, if both
  $E, T$ are $\trop^+_p$-relations, where $\trop^+_p$ is a semiring
  defined in Sec.~\ref{sec:pops}, then the program computes the {\em
    top $p+1$-shortest-paths}.
\end{ex}

The least fixpoint of $\name$, if exists, can be computed by the na\"ive
algorithm, quite similar to datalog: initialize all IDBs to $\bot$,
then repeatedly apply all rules of the $\name$ program, obtaining an
increasing chain of IDB instances,
$J^{(0)} \sqsubseteq J^{(1)} \sqsubseteq J^{(2)} \sqsubseteq \cdots$
When $J^{(t)} = J^{(t+1)}$, then the algorithm stops and returns
$J^{(t)}$; in that case we say that the $\name$ program converges in
$t$ steps, or we just say that it converges; otherwise we say that it
diverges.  Traditional datalog always converges, but this is no longer
the case for $\name$ programs.  There are five possibilities,
depending on the choice of the POPS $\bm P$:
%
\begin{enumerate}[label=(\roman*)]
\item \label{item:converge:1} For some $\name$ programs, $\bigvee_t J^{(t)}$ is not the least fixpoint.
\item \label{item:converge:2} Every $\name$ program has the least fixpoint $\bigvee_t J^{(t)}$, but may not necessarily converge.
\item \label{item:converge:3} Every $\name$ program converges.
\item \label{item:converge:4} Every $\name$ program converges in a
  number of steps that depends only on $|\adom(I)|$.
\item \label{item:converge:5} Every $\name$ program converges in a
  number of steps that is polynomial in $|\adom(I)|$.
\end{enumerate}
%
In this paper, we will only consider
data-complexity~\cite{DBLP:conf/stoc/Vardi82}, where the $\name$
program is assumed to be fixed, and the input consists only of the EDB
instance $I$.

We study algebraic properties of the POPS $\bm P$ that ensure that we
are in one of the cases~\ref{item:converge:3}-\ref{item:converge:5};
we do not address cases~\ref{item:converge:1}-\ref{item:converge:2} in
this paper.  We give next a necessary and sufficient condition for
each of the cases~\ref{item:converge:3} and~\ref{item:converge:4}, and
give a sufficient condition for case~\ref{item:converge:5}.  For any
POPS $\bm P$, the set
$\bm P \oplus \bot \defeq \setof{u\oplus \bot}{u \in P}$ is a
semiring (Proposition~\ref{prop:s:plus:bot}); our characterization is based entirely on a certain
property, called {\em stability}, of the semiring $\bm P \oplus \bot$,
which we describe here.

Given a semiring $\bm S$ and $u \in S$, denote by
$u^{(p)} := 1 \oplus u \oplus u^2 \oplus \cdots \oplus u^{p}$, where
$u^{i} := u \otimes u \otimes \cdots \otimes u$ ($i$ times).  We say
that $u$ is {\em $p$-stable} if $u^{(p)}=u^{(p+1)}$; we say that the
semiring $\bm S$ is {\em $p$-stable} if every element $u \in S$ is $p$-stable, and
we say that $\bm S$ is {\em stable} if every element $u$ is stable for some
$p$ that may depend on $u$.
A $\name$ program is {\em linear} if every rule has at most one IDB predicate in the body.
We prove:

\begin{thm} \label{th:main:intro} Given a POPS $\bm P$, the following hold.
  \begin{itemize}
  \item Every $\name$ program converges iff the semiring $\bm P\oplus \bot$ is stable.
  \item Every program converges in a number of steps that depends only
    on $|\adom(I)|$ iff $\bm P \oplus \bot$ is $p$-stable for some
    $p$.  More precisely, every $\name$ program converges in
    $\sum_{i=1}^{N}(p+2)^i$ steps, where $N$ is the number of ground
    tuples consisting of IDB predicates and constants from $\adom(I)$.
    Furthermore, if the program is linear, then it converges in $\sum_{i=1}^N(p+1)^i$ steps.
  \item If $\bm P\oplus \bot$ is $0$-stable, then every $\name$
    program converges in $N$ steps; in particular, the program runs in
    polynomial time in the size of the input database.
  \end{itemize}
\end{thm}

In a nutshell, the theorem says that convergence of $\name$ is
intimately related to the notion of stability.  The proof, provided in
Sec.~\ref{sec:complexity}, consists of an analysis of the infinite
powerseries resulting from unfolding the fixpoint definition; for
the proof of the first item we also use Parikh's
theorem~\cite{MR209093}.  As mentioned earlier, most prior work on
fixpoint equations assumes an $\omega$-continuous semiring; when
convergence is desired, one usually offers the Ascending Chain
Condition, ACC (see Sec.~\ref{sec:lfp}) as a sufficient condition for
convergence.  Our theorem implies that ACC is only a sufficient, but
not a necessary condition for convergence, for example $\trop^+$ is
$0$-stable, and therefore every $\name$ program converges on
$\trop^+$, yet it does not satisfy the ACC condition.  A somewhat
related result is proven by Luttenberger and
Schlund~\cite{DBLP:journals/iandc/LuttenbergerS16} who showed that, if
$1$ is $p$-stable, then Newton's method requires at most
$N + \log\log p$ iterations.  As mentioned earlier, each step of
Newton's method requires the computation of another least fixpoint,
hence that result does not inform us on the convergence of the na\"ive
algorithm.

Next, we introduce an extension of the semi-na\"ive evaluation
algorithm to $\name$.  It is known that the na\"ive evaluation
algorithm is inefficient in practice, because at each iteration it
repeats all the computations that it has done at the previous
iterations.  Most datalog systems implement an improvement called the
{\em semi-na\"ive} evaluation, which keeps track of the delta between
the successive states of the IDBs and applies the datalog rules as a
function of these deltas to obtain the new iteration's deltas.
Semi-na\"ive evaluation is one of the major optimization techniques
for evaluating datalog, however, it is defined only for programs that
are monotone under set inclusion, and the systems that implement it
enforce monotonicity, preventing the use of aggregation in
recursion. Our second result consists of showing how to adapt the
semi-na\"ive algorithm to $\name$, under certain restrictions of the
POPS $\bm P$, thus, enabling the semi-na\"ive algorithm to be applied
to programs with aggregation in recursion.

Let's illustrate the semi-na\"ive algorithm on the APSP program in
Example~\ref{ex:intro}.  For that, let's first review the standard
semi-na\"ive algorithm for pure datalog on the transitive closure
program~\eqref{eq:datalog:intro}.  While the na\"ive algorithm starts
with $T^{(0)}(X,Y)=\emptyset$ and repeats
$T^{(t+1)}(X,Y) = E(X,Y) \vee \bigvee_Z T^{(t)}(X,Z) \wedge E(Z,Y)$
for $t=0,1,2,\ldots$, the semi-na\"ive algorithm starts by
initializing $T^{(1)}(X,Y) = \delta^{(0)}(X,Y) = E(X,Y)$, then
performs the following steps for $t=1,2,3,\ldots$:
%
\begin{align}
  \delta^{(t)}(X,Y) &= \left(\bigvee_z \delta^{(t-1)}(X,Z) \wedge E(Z,Y) \right) \setminus T^{(t)}(X,Y) \label{eqn:deltat}\\
  T^{(t+1)}(X,Y) &= T^{(t)}(X,Y) \cup \delta^{(t)}(X,Y).\nonumber
\end{align}
%
Thus, the semi-na\"ive algorithm avoids re-discovering at iteration
$t$ tuples already discovered at iterations $1,2,\ldots, t-1$.

In order to apply the same principle to the APSP
program~\eqref{eqn:apsp}, we need to define an appropriate ``minus''
$\ominus$ operator on the tropical semiring.  This operator is:
%
\begin{align}
  v\ominus u=&
        \begin{cases} v &\mbox{if } v < u\\
          \infty &\mbox{if } v \geq u
        \end{cases} \label{eqn:trop:minus}
\end{align}
%
The semi-na\"ive algorithm for the APSP problem in~\eqref{eqn:linear:tc}
is a mirror of that in~\eqref{eqn:deltat}:
%
\begin{align}
  \delta^{(t)}(X,Y) &= (\min_Z \delta^{(t-1)}(X,Z) + E(Z,Y)) \ominus T^{(t)}(X,Y), \label{eq:semi-naive:trop}\\
  T^{(t+1)}(X,Y) &= \min(T^{(t)}(X,Y), \delta^{(t)}(X,Y))\nonumber
\end{align}
%
where the difference operator $\ominus$ is defined
in~\eqref{eqn:trop:minus}. The reason why this algorithm is more
efficient than the na\"ive algorithm is the fact that, in general, a
database system needs to store only the tuples that are ``present'' in
a relation, i.e. those whose value is $\neq \bot$.  In our example,
only those tuples $\delta^{(t)}(X,Y)$ whose value is $\neq \infty$
need to be stored.  The $\ominus$ operator
in~\eqref{eq:semi-naive:trop} checks if the new value
$\min_Z \delta^{(t-1)}(X,Z) + E(Z,Y)$ is strictly less than the old
value $T^{(t)}(X,Y)$: if not, then it returns $\infty$, signaling that
the value of $(X,Y)$ in both $\delta^{(t)}(X,Y)$ and $T^{(t)}(X,Y)$
does not need to be updated.  As a consequence, only those tuples
$T^{(t+1)}(X,Y)$ need to be processed at step $t$ where the value has
strictly decreased from the previous step.



Finally, the last topic we address in this paper is a possible way of
introducing negation in $\name$.  We will show that by interpreting
$\name$ over a particular POPS called THREE
(Sec.~\ref{subsec:three:pops}) and by appropriately defining a
function \texttt{not}, $\name$ can express datalog queries with
negation under Fitting's three-valued
semantics~\cite{DBLP:journals/jlp/Fitting85a}.  The crux in Fitting's
approach is to apply to the three-valued logic
(with truth values $\{0, 1,\bot\}$, where $\bot$ means ``undefined'')
the
{\em knowledge order} $\leq_k$ with $\bot \leq_k 0$ and
$\bot \leq_k 1$. Then, the function \texttt{not} defined as
$\mathtt{not}(0) = 1$, $\mathtt{not}(1) = 0$, and
$\mathtt{not}(\bot) = \bot$ is monotone w.r.t.\ $\leq_k$, and one can
apply the usual least fixpoint semantics to datalog programs with
negation (and, likewise, to $\name$ programs with the function
\texttt{not}).  Hence, in cases when Fitting's 3-valued semantics
coincides with the well-founded semantics, so does $\name$ equipped
with the function \texttt{not} when interpreted over the POPS THREE.

\subsection{Paper organization}

We define POPS in Sec.~\ref{sec:pops} and give several examples.
In Sec.~\ref{sec:lfp} we consider the least fixpoint of monotone
functions over posets, and prove an upper bound on the number of
iterations needed to compute the least fixpoint.  We define $\name$
formally in Sec.~\ref{sec:datalogo}, and give several examples.
The convergence results described in Theorem~\ref{th:main:intro} are
stated formally and proven in Sec.~\ref{sec:complexity}.
Sec.~\ref{sec:semi:naive} presents a generalization of semi-na\"ive
evaluation to $\name$.  We discuss how $\name$ can express datalog
queries with negation using 3-valued logic in
Sec.~\ref{sec:fitting}.  Finally, we conclude in
Sec.~\ref{sec:conclusions}.