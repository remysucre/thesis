% \newcommand{\missing}[1]{{\color{red}\bfseries [TODO]}}

% \newcommand{\etal}{et al.\xspace}

% \newcommand{\egg}{\texorpdfstring{\MakeLowercase{\texttt{egg}}}{\texttt{egg}}\xspace}
% \newcommand{\Egg}{\texorpdfstring{\MakeLowercase{\texttt{egg}}}{\texttt{egg}}\xspace}
% \newcommand{\egraphs}{\mbox{e-graphs}\xspace}
% \newcommand{\egraph}{\mbox{e-graph}\xspace}
% \newcommand{\Egraph}{\mbox{E-graph}\xspace}
% \newcommand{\Egraphs}{\mbox{E-graphs}\xspace}
% \newcommand{\eclass}{\mbox{e-class}\xspace}
% \newcommand{\Eclass}{\mbox{E-class}\xspace}
% \newcommand{\enode}{\mbox{e-node}\xspace}
% \newcommand{\eclasses}{\mbox{e-classes}\xspace}
% \newcommand{\enodes}{\mbox{e-nodes}\xspace}
% \newcommand{\Enodes}{\mbox{E-nodes}\xspace}
% % \newcommand{\regraph}{Regraph\xspace}
% % \newcommand{\Regraph}{Regraph\xspace}
% \newcommand{\sz}{Szalinski\xspace}
% \newcommand{\find}{\texttt{find}\xspace}

% \newcommand{\eqsat}{equality saturation\xspace}
% \newcommand{\Eqsat}{Equality saturation\xspace}

% \newcommand{\equivid}{\equiv_{\sf id}}
% \newcommand{\equivnode}{\equiv_{\sf node}}
% \newcommand{\equivterm}{\equiv_{\sf term}}

% \newcommand{\congrinv}{$\mathcal{I}_c$\xspace}

% \newcommand{\egglogo}[1][]{\protect\includegraphics[height=1em, #1]{egg.png} }
% \newcommand{\eggurl}{\url{https://github.com/mwillsey/egg}}

\newcommand\cse{Computer Science \& Engineering}
\newcommand\pgas{Paul G.\ Allen School of \cse}

% \newcommand{\CongrSpeedup}{\ensuremath{87.85\times}\xspace}
% \newcommand{\TotalSpeedup}{\ensuremath{20.96\times}\xspace}
% \newcommand{\RepairsR}{\ensuremath{0.98}\xspace}
% \newcommand{\RepairsP}{3.6e-47\xspace}
% \newcommand{\nEggTests}{32\xspace}
% \newcommand{\nEggTimeouts}{8\xspace}

%%% Local Variables:
%%% TeX-master: "thesis"
%%% End:

\usepackage{bm} % bold math fonts, e.g. $\bm \Sigma$ will give a bold-face $\Sigma$
% \usepackage{url}
% \usepackage{fullpage}

% \setlength{\marginparwidth}{2cm} % for todonotes
% \usepackage[colorinlistoftodos]{todonotes}

% % http://paultaylor.eu/diagrams/
% % https://www.jmilne.org/not/Mdiagrams.pdf
% \usepackage[small,nohug,heads=vee]{diagrams}
% \diagramstyle[labelstyle=\scriptstyle]

\usepackage[ruled, noend]{algorithm2e}
\usepackage{enumitem}
\usepackage{tikz}
\usetikzlibrary{shapes.geometric}
\usetikzlibrary{arrows.meta,arrows}
% \usepackage{quiver}
% \usepackage{subcaption}

%%% package microtype recommended by Paul Beame who says:
%%%%% try including [it] in any of your latex documents.  You will find that
%%%%% not only is the result more compact without changing any spacing
%%%%% parameters, the number of overfull/underfull warnings drops, and it
%%%%% simply looks a lot better!
%%%%%
%%%%% Why it works:  Latex is not known for the beauty of its typography.  The
%%%%% biggest issue is that the amount of visible white space between words
%%%%% varies quite a lot line-to-line in a paragraph since inter-word spacing
%%%%% is the only thing that will stretch/shrink.   The microtype package also
%%%%% makes imperceptible changes to the inter-letter spacing within words
%%%%% and, with all those extra degrees of freedom, it can do a much better
%%%%% job of laying out paragraphs.
% \usepackage{microtype}

\newcommand{\yell}[1]{{\color{red} \textbf{#1}}}

\newcommand{\rai}{Relational\underline{AI}}

\newcommand{\gv}[1]{\ensuremath{\mbox{\boldmath$ #1 $}}}
\newcommand{\grad}[1]{\gv{\nabla} #1}
\newcommand{\norm}[1]{\|#1\|}
\newcommand{\set}[1]{\{#1\}}                    % Set (as in \set{1,2,3}).
\newcommand{\bag}[1]{\{\hspace{-1mm}\{#1\}\hspace{-1mm}\}}                    % bag (as in \bag{1,2,3}).
\newcommand{\setof}[2]{\{{#1}\mid{#2}\}}        % Set (as in \setof{x}{x>0}).
\newcommand{\bagof}[2]{\{\hspace{-1mm}\{{#1}\mid{#2}\}\hspace{-1mm}\}}        % Set (as in \setof{x}{x>0}).
\newcommand{\pr}{\mathop{\textnormal{Prob}}}    % Probability
\newcommand{\E}{\mathop{\mathbb E}}    % Probability
\newcommand{\dom}{\textsf{Dom}}
\newcommand{\id}{\textsf{ID}}
\newcommand{\codom}{\textsf{CoDom}}
\newcommand{\one}{\bm 1}
\newcommand{\degree}{\text{\sf deg}}
\newcommand{\critdegree}{\text{\sf critDeg}}
\newcommand{\monomial}{\text{\sf Mon}}
\newcommand{\cmonomial}{\text{\sf \#Mon}}
\newcommand{\sign}{\text{\sf sign}}
\newcommand{\lfp}{\text{\sf lfp}}
\newcommand{\lpfp}{\text{\sf lpfp}}
\newcommand{\pfp}{\text{\sf pfp}}

\newcommand{\inner}[1]{\langle #1 \rangle}
%%
\newcommand{\LB}{\textsf{LogicBlox}}
\newcommand{\faq}{\textsf{FAQ}}
\newcommand{\calB}{\mathcal B}
\newcommand{\calC}{\mathcal C}
\newcommand{\calH}{\mathcal H}
\newcommand{\calV}{[n]}
\newcommand{\calE}{\mathcal E}
\newcommand{\calD}{\mathcal D}
\newcommand{\calW}{\mathcal W}
\newcommand{\calF}{\mathcal F}
\newcommand{\calP}{\mathcal P}
\newcommand{\calM}{\mathcal M}
\newcommand{\calN}{\mathcal N}
\newcommand{\calL}{\mathcal L}
\newcommand{\calS}{\mathcal S}
\newcommand{\calT}{\mathcal T}

% \theoremstyle{plain}                   % default
% \newtheorem{thm}{Theorem}[section]
% \newtheorem{lmm}[thm]{Lemma}
% \newtheorem{prop}[thm]{Proposition}
% \newtheorem{cor}[thm]{Corollary}
% \newtheorem{defn}[thm]{Definition}

% \theoremstyle{definition}              % Examples and all
% \newtheorem{pbm}{Problem}
% \newtheorem{opm}{Question}
% \newtheorem{conj}[thm]{Conjecture}
% \newtheorem{ex}[thm]{Example}
% \newtheorem{exer}{Exercise}
% \newtheorem{alg}[thm]{Algorithm}
% \newtheorem{rmk}[thm]{Remark}
% \newtheorem{claim}{Claim}
% \newtheorem{note}{Note}

\newcommand{\defeq}{\stackrel{\text{def}}{=}}
\newcommand{\mineq}{\stackrel{\text{min}}{=}}
\newcommand{\maxeq}{\stackrel{\text{max}}{=}}
\newcommand{\dleq}{\mbox{ :- }}
\newcommand{\mult}{\cdot}

\newcommand{\B}{\mathbb B} % the Booleans
\newcommand{\Z}{\mathbb Z} % integers
\newcommand{\N}{\mathbb N} % the natural numbers
\newcommand{\Q}{\mathbb Q} % the rational numbers
\newcommand{\R}{\mathbb R} % the real numbers
\newcommand{\T}{\mathbb T} % \set{\bot,\top}
%\newcommand{\C}{\mathbb C} % complex numbers
\newcommand{\D}{\mathbf D} % bold-face D, used for generic domain
% \newcommand{\U}{\mathbf U} % the universe


\newcommand{\cd}{\text{ :- }}
\newcommand{\iter}{\texttt{iter}}
\newcommand{\datalogo}{\texorpdfstring{$\text{Datalog}^\circ$\xspace}{Datalogo}}
\newcommand{\trop}{\text{\sf Trop}}
% \newcommand{\tropset}{\text{\sf T}}

\newcommand{\ground}{\textsf{GA}}
\newcommand{\arity}{\textsf{arity}}
\newcommand{\adom}{\textsf{ADom}}
\newcommand{\inst}{\textsf{Inst}}

\newcommand{\floor}[1]{\lfloor#1\rfloor}
\newcommand{\ceil}[1]{\lceil#1\rceil}
